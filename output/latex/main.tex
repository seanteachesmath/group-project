%********************************************%
%*       Generated from PreTeXt source      *%
%*       on 2021-04-09T14:51:43-05:00       *%
%*   A recent stable commit (2020-08-09):   *%
%* 98f21740783f166a773df4dc83cab5293ab63a4a *%
%*                                          *%
%*         https://pretextbook.org          *%
%*                                          *%
%********************************************%
\documentclass[oneside,10pt,]{book}
\usepackage{axessibility}

%% Custom Preamble Entries, early (use latex.preamble.early)
%% Default LaTeX packages
%%   1.  always employed (or nearly so) for some purpose, or
%%   2.  a stylewriter may assume their presence
\usepackage{geometry}
%% Some aspects of the preamble are conditional,
%% the LaTeX engine is one such determinant
\usepackage{ifthen}
%% etoolbox has a variety of modern conveniences
\usepackage{etoolbox}
\usepackage{ifxetex,ifluatex}
%% Raster graphics inclusion
\usepackage{graphicx}
%% Color support, xcolor package
%% Always loaded, for: add/delete text, author tools
%% Here, since tcolorbox loads tikz, and tikz loads xcolor
\PassOptionsToPackage{usenames,dvipsnames,svgnames,table}{xcolor}
\usepackage{xcolor}
%% begin: defined colors, via xcolor package, for styling
%% end: defined colors, via xcolor package, for styling
%% Colored boxes, and much more, though mostly styling
%% skins library provides "enhanced" skin, employing tikzpicture
%% boxes may be configured as "breakable" or "unbreakable"
%% "raster" controls grids of boxes, aka side-by-side
\usepackage{tcolorbox}
\tcbuselibrary{skins}
\tcbuselibrary{breakable}
\tcbuselibrary{raster}
%% We load some "stock" tcolorbox styles that we use a lot
%% Placement here is provisional, there will be some color work also
%% First, black on white, no border, transparent, but no assumption about titles
\tcbset{ bwminimalstyle/.style={size=minimal, boxrule=-0.3pt, frame empty,
colback=white, colbacktitle=white, coltitle=black, opacityfill=0.0} }
%% Second, bold title, run-in to text/paragraph/heading
%% Space afterwards will be controlled by environment,
%% independent of constructions of the tcb title
%% Places \blocktitlefont onto many block titles
\tcbset{ runintitlestyle/.style={fonttitle=\blocktitlefont\upshape\bfseries, attach title to upper} }
%% Spacing prior to each exercise, anywhere
\tcbset{ exercisespacingstyle/.style={before skip={1.5ex plus 0.5ex}} }
%% Spacing prior to each block
\tcbset{ blockspacingstyle/.style={before skip={2.0ex plus 0.5ex}} }
%% xparse allows the construction of more robust commands,
%% this is a necessity for isolating styling and behavior
%% The tcolorbox library of the same name loads the base library
\tcbuselibrary{xparse}
%% Hyperref should be here, but likes to be loaded late
%%
%% Inline math delimiters, \(, \), need to be robust
%% 2016-01-31:  latexrelease.sty  supersedes  fixltx2e.sty
%% If  latexrelease.sty  exists, bugfix is in kernel
%% If not, bugfix is in  fixltx2e.sty
%% See:  https://tug.org/TUGboat/tb36-3/tb114ltnews22.pdf
%% and read "Fewer fragile commands" in distribution's  latexchanges.pdf
\IfFileExists{latexrelease.sty}{}{\usepackage{fixltx2e}}
%% Text height identically 9 inches, text width varies on point size
%% See Bringhurst 2.1.1 on measure for recommendations
%% 75 characters per line (count spaces, punctuation) is target
%% which is the upper limit of Bringhurst's recommendations
\geometry{letterpaper,total={340pt,9.0in}}
%% Custom Page Layout Adjustments (use latex.geometry)
%% This LaTeX file may be compiled with pdflatex, xelatex, or lualatex executables
%% LuaTeX is not explicitly supported, but we do accept additions from knowledgeable users
%% The conditional below provides  pdflatex  specific configuration last
%% begin: engine-specific capabilities
\ifthenelse{\boolean{xetex} \or \boolean{luatex}}{%
%% begin: xelatex and lualatex-specific default configuration
\ifxetex\usepackage{xltxtra}\fi
%% realscripts is the only part of xltxtra relevant to lualatex 
\ifluatex\usepackage{realscripts}\fi
%% end:   xelatex and lualatex-specific default configuration
}{
%% begin: pdflatex-specific default configuration
%% We assume a PreTeXt XML source file may have Unicode characters
%% and so we ask LaTeX to parse a UTF-8 encoded file
%% This may work well for accented characters in Western language,
%% but not with Greek, Asian languages, etc.
%% When this is not good enough, switch to the  xelatex  engine
%% where Unicode is better supported (encouraged, even)
\usepackage[utf8]{inputenc}
%% end: pdflatex-specific default configuration
}
%% end:   engine-specific capabilities
%%
%% Fonts.  Conditional on LaTex engine employed.
%% Default Text Font: The Latin Modern fonts are
%% "enhanced versions of the [original TeX] Computer Modern fonts."
%% We use them as the default text font for PreTeXt output.
%% Default Monospace font: Inconsolata (aka zi4)
%% Sponsored by TUG: http://levien.com/type/myfonts/inconsolata.html
%% Loaded for documents with intentional objects requiring monospace
%% See package documentation for excellent instructions
%% fontspec will work universally if we use filename to locate OTF files
%% Loads the "upquote" package as needed, so we don't have to
%% Upright quotes might come from the  textcomp  package, which we also use
%% We employ the shapely \ell to match Google Font version
%% pdflatex: "varl" package option produces shapely \ell
%% pdflatex: "var0" package option produces plain zero (not used)
%% pdflatex: "varqu" package option produces best upright quotes
%% xelatex,lualatex: add OTF StylisticSet 1 for shapely \ell
%% xelatex,lualatex: add OTF StylisticSet 2 for plain zero (not used)
%% xelatex,lualatex: add OTF StylisticSet 3 for upright quotes
%%
%% Automatic Font Control
%% Portions of a document, are, or may, be affected by defined commands
%% These are perhaps more flexible when using  xelatex  rather than  pdflatex
%% The following definitions are meant to be re-defined in a style, using \renewcommand
%% They are scoped when employed (in a TeX group), and so should not be defined with an argument
\newcommand{\divisionfont}{\relax}
\newcommand{\blocktitlefont}{\relax}
\newcommand{\contentsfont}{\relax}
\newcommand{\pagefont}{\relax}
\newcommand{\tabularfont}{\relax}
\newcommand{\xreffont}{\relax}
\newcommand{\titlepagefont}{\relax}
%%
\ifthenelse{\boolean{xetex} \or \boolean{luatex}}{%
%% begin: font setup and configuration for use with xelatex
%% Generally, xelatex is necessary for non-Western fonts
%% fontspec package provides extensive control of system fonts,
%% meaning *.otf (OpenType), and apparently *.ttf (TrueType)
%% that live *outside* your TeX/MF tree, and are controlled by your *system*
%% (it is possible that a TeX distribution will place fonts in a system location)
%%
%% The fontspec package is the best vehicle for using different fonts in  xelatex
%% So we load it always, no matter what a publisher or style might want
%%
\usepackage{fontspec}
%%
%% begin: xelatex main font ("font-xelatex-main" template)
%% Latin Modern Roman is the default font for xelatex and so is loaded with a TU encoding
%% *in the format* so we can't touch it, only perhaps adjust it later
%% in one of two ways (then known by NFSS names such as "lmr")
%% (1) via NFSS with font family names such as "lmr" and "lmss"
%% (2) via fontspec with commands like \setmainfont{Latin Modern Roman}
%% The latter requires the font to be known at the system-level by its font name,
%% but will give access to OTF font features through optional arguments
%% https://tex.stackexchange.com/questions/470008/
%% where-and-how-does-fontspec-sty-specify-the-default-font-latin-modern-roman
%% http://tex.stackexchange.com/questions/115321
%% /how-to-optimize-latin-modern-font-with-xelatex
%%
%% end:   xelatex main font ("font-xelatex-main" template)
%% begin: xelatex mono font ("font-xelatex-mono" template)
%% (conditional on non-trivial uses being present in source)
\IfFontExistsTF{Inconsolatazi4-Regular.otf}{}{\GenericError{}{The font "Inconsolatazi4-Regular.otf" requested by PreTeXt output is not available.  Either a file cannot be located in default locations via a filename, or a font is not known by its name as part of your system.}{Consult the PreTeXt Guide for help with LaTeX fonts.}{}}
\IfFontExistsTF{Inconsolatazi4-Bold.otf}{}{\GenericError{}{The font "Inconsolatazi4-Bold.otf" requested by PreTeXt output is not available.  Either a file cannot be located in default locations via a filename, or a font is not known by its name as part of your system.}{Consult the PreTeXt Guide for help with LaTeX fonts.}{}}
\usepackage{zi4}
\setmonofont[BoldFont=Inconsolatazi4-Bold.otf,StylisticSet={1,3}]{Inconsolatazi4-Regular.otf}
%% end:   xelatex mono font ("font-xelatex-mono" template)
%% begin: xelatex font adjustments ("font-xelatex-style" template)
%% end:   xelatex font adjustments ("font-xelatex-style" template)
%%
%% Extensive support for other languages
\usepackage{polyglossia}
%% Set main/default language based on pretext/@xml:lang value
%% document language code is "en-US", US English
%% usmax variant has extra hypenation
\setmainlanguage[variant=usmax]{english}
%% Enable secondary languages based on discovery of @xml:lang values
%% Enable fonts/scripts based on discovery of @xml:lang values
%% Western languages should be ably covered by Latin Modern Roman
%% end:   font setup and configuration for use with xelatex
}{%
%% begin: font setup and configuration for use with pdflatex
%% begin: pdflatex main font ("font-pdflatex-main" template)
\usepackage{lmodern}
\usepackage[T1]{fontenc}
%% end:   pdflatex main font ("font-pdflatex-main" template)
%% begin: pdflatex mono font ("font-pdflatex-mono" template)
%% (conditional on non-trivial uses being present in source)
\usepackage[varqu,varl]{inconsolata}
%% end:   pdflatex mono font ("font-pdflatex-mono" template)
%% begin: pdflatex font adjustments ("font-pdflatex-style" template)
%% end:   pdflatex font adjustments ("font-pdflatex-style" template)
%% end:   font setup and configuration for use with pdflatex
}
%% Symbols, align environment, commutative diagrams, bracket-matrix
\usepackage{amsmath}
\usepackage{amscd}
\usepackage{amssymb}
%% allow page breaks within display mathematics anywhere
%% level 4 is maximally permissive
%% this is exactly the opposite of AMSmath package philosophy
%% there are per-display, and per-equation options to control this
%% split, aligned, gathered, and alignedat are not affected
\allowdisplaybreaks[4]
%% allow more columns to a matrix
%% can make this even bigger by overriding with  latex.preamble.late  processing option
\setcounter{MaxMatrixCols}{30}
%%
%%
%% Division Titles, and Page Headers/Footers
%% titlesec package, loading "titleps" package cooperatively
%% See code comments about the necessity and purpose of "explicit" option.
%% The "newparttoc" option causes a consistent entry for parts in the ToC 
%% file, but it is only effective if there is a \titleformat for \part.
%% "pagestyles" loads the  titleps  package cooperatively.
\usepackage[explicit, newparttoc, pagestyles]{titlesec}
%% The companion titletoc package for the ToC.
\usepackage{titletoc}
%% Fixes a bug with transition from chapters to appendices in a "book"
%% See generating XSL code for more details about necessity
\newtitlemark{\chaptertitlename}
%% begin: customizations of page styles via the modal "titleps-style" template
%% Designed to use commands from the LaTeX "titleps" package
%% Plain pages should have the same font for page numbers
\renewpagestyle{plain}{%
\setfoot{}{\pagefont\thepage}{}%
}%
%% Single pages as in default LaTeX
\renewpagestyle{headings}{%
\sethead{\pagefont\slshape\MakeUppercase{\ifthechapter{\chaptertitlename\space\thechapter.\space}{}\chaptertitle}}{}{\pagefont\thepage}%
}%
\pagestyle{headings}
%% end: customizations of page styles via the modal "titleps-style" template
%%
%% Create globally-available macros to be provided for style writers
%% These are redefined for each occurence of each division
\newcommand{\divisionnameptx}{\relax}%
\newcommand{\titleptx}{\relax}%
\newcommand{\subtitleptx}{\relax}%
\newcommand{\shortitleptx}{\relax}%
\newcommand{\authorsptx}{\relax}%
\newcommand{\epigraphptx}{\relax}%
%% Create environments for possible occurences of each division
%% Environment for a PTX "chapter" at the level of a LaTeX "chapter"
\NewDocumentEnvironment{chapterptx}{mmmmmm}
{%
\renewcommand{\divisionnameptx}{Chapter}%
\renewcommand{\titleptx}{#1}%
\renewcommand{\subtitleptx}{#2}%
\renewcommand{\shortitleptx}{#3}%
\renewcommand{\authorsptx}{#4}%
\renewcommand{\epigraphptx}{#5}%
\chapter[{#3}]{#1}%
\label{#6}%
}{}%
%% Environment for a PTX "section" at the level of a LaTeX "section"
\NewDocumentEnvironment{sectionptx}{mmmmmm}
{%
\renewcommand{\divisionnameptx}{Section}%
\renewcommand{\titleptx}{#1}%
\renewcommand{\subtitleptx}{#2}%
\renewcommand{\shortitleptx}{#3}%
\renewcommand{\authorsptx}{#4}%
\renewcommand{\epigraphptx}{#5}%
\section[{#3}]{#1}%
\label{#6}%
}{}%
%% Environment for a PTX "subsection" at the level of a LaTeX "subsection"
\NewDocumentEnvironment{subsectionptx}{mmmmmm}
{%
\renewcommand{\divisionnameptx}{Subsection}%
\renewcommand{\titleptx}{#1}%
\renewcommand{\subtitleptx}{#2}%
\renewcommand{\shortitleptx}{#3}%
\renewcommand{\authorsptx}{#4}%
\renewcommand{\epigraphptx}{#5}%
\subsection[{#3}]{#1}%
\label{#6}%
}{}%
%%
%% Styles for six traditional LaTeX divisions
\titleformat{\part}[display]
{\divisionfont\Huge\bfseries\centering}{\divisionnameptx\space\thepart}{30pt}{\Huge#1}
[{\Large\centering\authorsptx}]
\titleformat{\chapter}[display]
{\divisionfont\huge\bfseries}{\divisionnameptx\space\thechapter}{20pt}{\Huge#1}
[{\Large\authorsptx}]
\titleformat{name=\chapter,numberless}[display]
{\divisionfont\huge\bfseries}{}{0pt}{#1}
[{\Large\authorsptx}]
\titlespacing*{\chapter}{0pt}{50pt}{40pt}
\titleformat{\section}[hang]
{\divisionfont\Large\bfseries}{\thesection}{1ex}{#1}
[{\large\authorsptx}]
\titleformat{name=\section,numberless}[block]
{\divisionfont\Large\bfseries}{}{0pt}{#1}
[{\large\authorsptx}]
\titlespacing*{\section}{0pt}{3.5ex plus 1ex minus .2ex}{2.3ex plus .2ex}
\titleformat{\subsection}[hang]
{\divisionfont\large\bfseries}{\thesubsection}{1ex}{#1}
[{\normalsize\authorsptx}]
\titleformat{name=\subsection,numberless}[block]
{\divisionfont\large\bfseries}{}{0pt}{#1}
[{\normalsize\authorsptx}]
\titlespacing*{\subsection}{0pt}{3.25ex plus 1ex minus .2ex}{1.5ex plus .2ex}
\titleformat{\subsubsection}[hang]
{\divisionfont\normalsize\bfseries}{\thesubsubsection}{1em}{#1}
[{\small\authorsptx}]
\titleformat{name=\subsubsection,numberless}[block]
{\divisionfont\normalsize\bfseries}{}{0pt}{#1}
[{\normalsize\authorsptx}]
\titlespacing*{\subsubsection}{0pt}{3.25ex plus 1ex minus .2ex}{1.5ex plus .2ex}
\titleformat{\paragraph}[hang]
{\divisionfont\normalsize\bfseries}{\theparagraph}{1em}{#1}
[{\small\authorsptx}]
\titleformat{name=\paragraph,numberless}[block]
{\divisionfont\normalsize\bfseries}{}{0pt}{#1}
[{\normalsize\authorsptx}]
\titlespacing*{\paragraph}{0pt}{3.25ex plus 1ex minus .2ex}{1.5em}
%%
%% Styles for five traditional LaTeX divisions
\titlecontents{part}%
[0pt]{\contentsmargin{0em}\addvspace{1pc}\contentsfont\bfseries}%
{\Large\thecontentslabel\enspace}{\Large}%
{}%
[\addvspace{.5pc}]%
\titlecontents{chapter}%
[0pt]{\contentsmargin{0em}\addvspace{1pc}\contentsfont\bfseries}%
{\large\thecontentslabel\enspace}{\large}%
{\hfill\bfseries\thecontentspage}%
[\addvspace{.5pc}]%
\dottedcontents{section}[3.8em]{\contentsfont}{2.3em}{1pc}%
\dottedcontents{subsection}[6.1em]{\contentsfont}{3.2em}{1pc}%
\dottedcontents{subsubsection}[9.3em]{\contentsfont}{4.3em}{1pc}%
%%
%% Begin: Semantic Macros
%% To preserve meaning in a LaTeX file
%%
%% \mono macro for content of "c", "cd", "tag", etc elements
%% Also used automatically in other constructions
%% Simply an alias for \texttt
%% Always defined, even if there is no need, or if a specific tt font is not loaded
\newcommand{\mono}[1]{\texttt{#1}}
%%
%% Following semantic macros are only defined here if their
%% use is required only in this specific document
%%
%% Used for inline definitions of terms
\newcommand{\terminology}[1]{\textbf{#1}}
%% End: Semantic Macros
%% Division Numbering: Chapters, Sections, Subsections, etc
%% Division numbers may be turned off at some level ("depth")
%% A section *always* has depth 1, contrary to us counting from the document root
%% The latex default is 3.  If a larger number is present here, then
%% removing this command may make some cross-references ambiguous
%% The precursor variable $numbering-maxlevel is checked for consistency in the common XSL file
\setcounter{secnumdepth}{3}
%%
%% AMS "proof" environment is no longer used, but we leave previously
%% implemented \qedhere in place, should the LaTeX be recycled
\newcommand{\qedhere}{\relax}
%%
%% A faux tcolorbox whose only purpose is to provide common numbering
%% facilities for most blocks (possibly not projects, 2D displays)
%% Controlled by  numbering.theorems.level  processing parameter
\newtcolorbox[auto counter, number within=section]{block}{}
%%
%% This document is set to number PROJECT-LIKE on a separate numbering scheme
%% So, a faux tcolorbox whose only purpose is to provide this numbering
%% Controlled by  numbering.projects.level  processing parameter
\newtcolorbox[auto counter, number within=section]{project-distinct}{}
%% A faux tcolorbox whose only purpose is to provide common numbering
%% facilities for 2D displays which are subnumbered as part of a "sidebyside"
\makeatletter
\newtcolorbox[auto counter, number within=tcb@cnt@block, number freestyle={\noexpand\thetcb@cnt@block(\noexpand\alph{\tcbcounter})}]{subdisplay}{}
\makeatother
%%
%% tcolorbox, with styles, for DEFINITION-LIKE
%%
%% definition: fairly simple numbered block/structure
\tcbset{ definitionstyle/.style={bwminimalstyle, runintitlestyle, blockspacingstyle, after title={\space}, after upper={\space\space\hspace*{\stretch{1}}\(\lozenge\)}, } }
\newtcolorbox[use counter from=block]{definition}[2]{title={{Definition~\thetcbcounter\notblank{#1}{\space\space#1}{}}}, phantomlabel={#2}, breakable, parbox=false, after={\par}, definitionstyle, }
%%
%% tcolorbox, with styles, for REMARK-LIKE
%%
%% note: fairly simple numbered block/structure
\tcbset{ notestyle/.style={bwminimalstyle, runintitlestyle, blockspacingstyle, after title={\space}, } }
\newtcolorbox[use counter from=block]{note}[2]{title={{Note~\thetcbcounter\notblank{#1}{\space\space#1}{}}}, phantomlabel={#2}, breakable, parbox=false, after={\par}, notestyle, }
%%
%% tcolorbox, with styles, for EXAMPLE-LIKE
%%
%% example: fairly simple numbered block/structure
\tcbset{ examplestyle/.style={bwminimalstyle, runintitlestyle, blockspacingstyle, after title={\space}, after upper={\space\space\hspace*{\stretch{1}}\(\square\)}, } }
\newtcolorbox[use counter from=block]{example}[2]{title={{Example~\thetcbcounter\notblank{#1}{\space\space#1}{}}}, phantomlabel={#2}, breakable, parbox=false, after={\par}, examplestyle, }
%%
%% tcolorbox, with styles, for inline exercises
%%
%% inlineexercise: fairly simple numbered block/structure
\tcbset{ inlineexercisestyle/.style={bwminimalstyle, runintitlestyle, blockspacingstyle, after title={\space}, } }
\newtcolorbox[use counter from=block]{inlineexercise}[2]{title={{Checkpoint~\thetcbcounter\notblank{#1}{\space\space#1}{}}}, phantomlabel={#2}, breakable, parbox=false, after={\par}, inlineexercisestyle, }
%%
%% xparse environments for introductions and conclusions of divisions
%%
%% introduction: in a structured division
\NewDocumentEnvironment{introduction}{m}
{\notblank{#1}{\noindent\textbf{#1}\space}{}}{\par\medskip}
%% conclusion: in a structured division
\NewDocumentEnvironment{conclusion}{m}
{\par\medskip\noindent\notblank{#1}{\textbf{#1}\space}{}}{}
%%
%% tcolorbox, with styles, for miscellaneous environments
%%
%% paragraphs: the terminal, pseudo-division
%% We use the lowest LaTeX traditional division
\titleformat{\subparagraph}[runin]{\normalfont\normalsize\bfseries}{\thesubparagraph}{1em}{#1}
\titlespacing*{\subparagraph}{0pt}{3.25ex plus 1ex minus .2ex}{1em}
\NewDocumentEnvironment{paragraphs}{mm}
{\subparagraph*{#1}\hypertarget{#2}{}}{}
%% Localize LaTeX supplied names (possibly none)
\renewcommand*{\chaptername}{Chapter}
%% Equation Numbering
%% Controlled by  numbering.equations.level  processing parameter
%% No adjustment here implies document-wide numbering
\numberwithin{equation}{section}
%% More flexible list management, esp. for references
%% But also for specifying labels (i.e. custom order) on nested lists
\usepackage{enumitem}
%% hyperref driver does not need to be specified, it will be detected
%% Footnote marks in tcolorbox have broken linking under
%% hyperref, so it is necessary to turn off all linking
%% It *must* be given as a package option, not with \hypersetup
\usepackage[pdftitle=Book,pdflang=en-US, hyperfootnotes=false]{hyperref}
%% configure hyperref's  \url  to match listings' inline verbatim
\renewcommand\UrlFont{\small\ttfamily}
%% Hyperlinking active in electronic PDFs, all links solid and blue
\hypersetup{colorlinks=true,linkcolor=blue,citecolor=blue,filecolor=blue,urlcolor=blue}
\hypersetup{pdftitle={Group Project}}
%% If you manually remove hyperref, leave in this next command
\providecommand\phantomsection{}
%% Graphics Preamble Entries
\usepackage{tikz}
%% If tikz has been loaded, replace ampersand with \amp macro
%% extpfeil package for certain extensible arrows,
%% as also provided by MathJax extension of the same name
%% NB: this package loads mtools, which loads calc, which redefines
%%     \setlength, so it can be removed if it seems to be in the 
%%     way and your math does not use:
%%     
%%     \xtwoheadrightarrow, \xtwoheadleftarrow, \xmapsto, \xlongequal, \xtofrom
%%     
%%     we have had to be extra careful with variable thickness
%%     lines in tables, and so also load this package late
\usepackage{extpfeil}
%% Custom Preamble Entries, late (use latex.preamble.late)
%% Begin: Author-provided packages
%% (From  docinfo/latex-preamble/package  elements)
%% End: Author-provided packages
%% Begin: Author-provided macros
%% (From  docinfo/macros  element)
%% Plus three from MBX for XML characters
\newcommand{\foo}{bar}
\DeclareMathOperator{\csch}{csch}
\newcommand{\lt}{<}
\newcommand{\gt}{>}
\newcommand{\amp}{&}
%% End: Author-provided macros
\begin{document}
%
%
\typeout{************************************************}
\typeout{Chapter 1 My First Chapter}
\typeout{************************************************}
%
\begin{chapterptx}{My First Chapter}{}{My First Chapter}{}{}{x:chapter:my-first-chapter}
%
%
\typeout{************************************************}
\typeout{Section 1.1 First section}
\typeout{************************************************}
%
\begin{sectionptx}{First section}{}{First section}{}{}{x:section:section-sample}
This is a short sentence. This is an equation shown in the sentence \(a^2 + b^2 = c^2\). A more fancy presentation is%
\begin{equation*}
x =
\dfrac{-b \pm \sqrt{b^2-4ac}}{2a}.
\end{equation*}
%
\begin{definition}{tag.}{x:definition:def-tag}%
This content was created with the \terminology{tag} for \mono{definition}.  Also in use are the \terminology{tags} for \mono{title}, \mono{statement}, \mono{term}, and \mono{c}.\end{definition}
%
%
\typeout{************************************************}
\typeout{Section 1.1.1 Aligned content}
\typeout{************************************************}
%
\begin{sectionptx}{Aligned content}{}{Aligned content}{}{}{x:section:sub-align}
To align equations, check out the following code,%
\begin{equation*}
\begin{aligned}
(x+1)^2 &= (x+1)(x+1)\\
&= x^2+x+x+1\\
&= x^2 + 2x +1
\end{aligned}
\end{equation*}
Or, perhaps better yet,%
\begin{align*}
(x+1)^2 &= (x+1)(x+1)\\
&= x^2+x+x+1\\
&= x^2 + 2x +1
\end{align*}
This is the hyperbolic cosecant function, \(\csch(x)\). %
\end{sectionptx}
\end{sectionptx}
%
%
\typeout{************************************************}
\typeout{Section 1.2 Sean's section}
\typeout{************************************************}
%
\begin{sectionptx}{Sean's section}{}{Sean's section}{}{}{x:section:sec-unique-keyword-sean}
%
%
\typeout{************************************************}
\typeout{Subsection 1.2.1 Sample subsection}
\typeout{************************************************}
%
\begin{subsectionptx}{Sample subsection}{}{Sample subsection}{}{}{x:subsection:sub-sec-reference-sean}
\begin{paragraphs}{Paragraph heading.}{x:paragraphs:paragraph-sean}%
\end{paragraphs}%
\end{subsectionptx}
\end{sectionptx}
%
%
\typeout{************************************************}
\typeout{Section 1.3 Sample section}
\typeout{************************************************}
%
\begin{sectionptx}{Sample section}{}{Sample section}{}{}{x:section:sec-unique-keyword-devon}
%
%
\typeout{************************************************}
\typeout{Subsection 1.3.1 Sample subsection}
\typeout{************************************************}
%
\begin{subsectionptx}{Sample subsection}{}{Sample subsection}{}{}{x:subsection:sub-sec-reference-devon}
\begin{paragraphs}{Paragraph heading.}{x:paragraphs:sample-devon}%
\end{paragraphs}%
\end{subsectionptx}
\end{sectionptx}
%
%
\typeout{************************************************}
\typeout{Section 1.4 Eli's Section}
\typeout{************************************************}
%
\begin{sectionptx}{Eli's Section}{}{Eli's Section}{}{}{x:section:sec-unique-keyword-eli}
\begin{introduction}{}%
This section is dedicated to my work, discoveries, mishaps, and progressions made within PreTeXt. It is broken up into subsections of different types of material. Hopefully it is set up in an easily readable manner.%
\par
Back to my \href{https://EsromGile.github.io}{github.io}.%
\begin{note}{}{x:note:note-tag}%
I'm open to any suggestions on the format of my section\slash{}subsections.%
\end{note}
\end{introduction}%
%
%
\typeout{************************************************}
\typeout{Subsection 1.4.1 Getting Started}
\typeout{************************************************}
%
\begin{subsectionptx}{Getting Started}{}{Getting Started}{}{}{x:subsection:sub-getting-started}
\begin{inlineexercise}{Setting Up.}{x:exercise:check-setting-up}%
Getting started was one of the easier parts for me using a Linux operating system. Following the guides on \href{https://pretextbook.org/}{PreText}, \href{https://pretextbook.github.io/pretext-cli/pretext-cli-documentation.html}{PreTeXt CLI}, and \href{https://pretextbook.org/doc/guide/html/guide-toc.html}{PreTeXt Guide} were sufficient material and didn't take but an hour to successfully install an have everything up and running.%
\par
The only issue for me on Linux is that I had to remove some double quotes and backslashes from the \href{https://pretextbook.github.io/pretext-cli/pretext-cli-documentation.html}{PreTeXt CLI} guide when adding python to PATH.%
\end{inlineexercise}
\begin{inlineexercise}{Next Steps.}{x:exercise:check-next-steps}%
1.) After getting started, the next step is to begin undertaking your first book. To begin choose a directory you want your project to be in, then in the terminal type 'pretext new "name\textunderscore{}of\textunderscore{}your\textunderscore{}book"'. Move into the newly created directory (same name as your book) and type 'pretext build' for it to fully create the output files.%
\par
2.) Then the fun begins! To create your first section, use the '\textless{}section	\textgreater{} and \textless{}\slash{}section\textgreater{}' tags. Type the desired contents inbetween these two tags to to write the section. When you create a sectionm, it is also good to add an 'xml:id' to the tag. This is inserted in the '\textless{}section	\textgreater{}' tag. Ex: \mono{<section xml:id=\textquotedbl{}sec-example\textquotedbl{}>}%
\par
3.) Add a title to your section by using the '\textless{}title\textgreater{} and \textless{}\slash{}title\textgreater{}' tags udnerneath the section you would like to define. Just place the desired name of the title between the two tags as shown. Ex: \mono{<title> Example Title </title>}%
\begin{note}{}{g:note:idp140347823423680}%
Steps 2 and 3 can be done with most any tag.\end{note}
\end{inlineexercise}
\end{subsectionptx}
%
%
\typeout{************************************************}
\typeout{Subsection 1.4.2 General}
\typeout{************************************************}
%
\begin{subsectionptx}{General}{}{General}{}{}{x:subsection:sub-general}
\begin{inlineexercise}{Math Notation.}{x:exercise:check-math-notation}%
Read more about math notaion \href{https://pretextbook.org/doc/guide/html/topic-mathematics.html}{here}.%
\par
A short list of \mono{General Tags} and how they are used:%
\begin{itemize}[label=\textbullet]
\item{}\textless{}m\textgreater{} is for inserting math notation \emph{inline}%
\item{}\textless{}me\textgreater{} is used for inseting math notation \emph{on it's own line}%
\item{}\textless{}md\textgreater{} is used for writing out a problem with the \emph{equal signs lined up}%
\item{}\textless{}mrow\textgreater{} is used inside the \textless{}md\textgreater{} tag to distinguish each \emph{row of the equation} \begin{note}{}{g:note:idp140347821895424}%
When you use the \textless{}mrow\textgreater{} tag, the equal signs you want lined up must be preceded by \textbackslash{}amp, ex: ``\textbackslash{}amp =''.\end{note}
%
\end{itemize}
%
\par
A short list of \mono{Arithmatic Tags} (these tags must be used inside of \mono{General Tags}):%
\begin{itemize}[label=\textbullet]
\item{}\textbackslash{}int is how to write \emph{indefinite integrals}%
\item{}\textbackslash{}int\textunderscore{}a\textasciicircum{}b is how to write \emph{definite integrals}%
\item{}\textbackslash{}lim\textunderscore{}\textbraceleft{}n\textbackslash{}to\textbackslash{}infty\textbraceright{} is how to write \emph{limits to inifinity} ("n" and "\textbackslash{}infinity" can be replaced by any parameters you want to take).%
\item{}\textbackslash{}sum is how to get the \(\Sigma\) for \emph{sumation notaion}, but for it to work correctly you need to follow it with one of two tags:%
\begin{itemize}[label=$\circ$]
\item{}\textbackslash{}limits\textunderscore{}\textbraceleft{}i=1\textbraceright{}\textasciicircum{}n%
\item{}\textbackslash{}nolimits\textunderscore{}\textbraceleft{}subscript\textbraceright{}%
\end{itemize}
\begin{note}{}{g:note:idp140347824292816}%
With \textbackslash{}int, \textbackslash{}lim\textunderscore{}\textbraceleft{}n\textbackslash{}to\textbackslash{}infty\textbraceright{} or \textbackslash{}sum, if you want their sub and superscripts to appear above and below the symbol, you need to precede them with \textbackslash{}displaystyle, ex: "\textbackslash{}displaystyle\textbackslash{}lim\textunderscore{}\textbraceleft{}n\textbackslash{}to\textbackslash{}infty\textbraceright{}" = \(\displaystyle\lim_{n\to\infty}\)\end{note}
%
\item{}Greek letters can be written with "\textbackslash{}" followed by the name of the letter, ex: "\textbackslash{}Sigma, \textbackslash{}sigma" = \(\Sigma\), \(\sigma\)%
\end{itemize}
%
\end{inlineexercise}
\begin{inlineexercise}{Lists.}{x:exercise:check-lists}%
Read more about lists \href{https://pretextbook.org/doc/guide/html/topic-lists.html}{here}.%
\par
There are three main types of lists available:%
\begin{itemize}[label=\textbullet]
\item{}unordered lists which use the \textless{}ul\textgreater{} tag%
\item{}ordered lists which use the \textless{}ol\textgreater{} tag%
\item{}description lists which use the \textless{}dl\textgreater{} tag%
\end{itemize}
For each item\slash{}row, in any of the above, use the list item tag, \textless{}li\textgreater{}, followed by the paragraph tag \textless{}p\textgreater{} and place the contents inside.%
\end{inlineexercise}
\begin{inlineexercise}{Divisions.}{x:exercise:check-subsections}%
Read more about subsections and documents divisions \href{https://pretextbook.org/doc/guide/html/topic-divisions.html}{here}.%
\par
The three basic divisions inside of each pretext book are going to be the \textless{}chapter\textgreater{}, \textless{}section\textgreater{}, and \textless{}subsection\textgreater{}. In my experience, when using the \textless{}subsection\textgreater{} tag, a \textless{}introduction\textgreater{} and a \textless{}conclusion\textgreater{} must be included before and after any subsection in order for the compiler not to complain. \begin{note}{}{g:note:idp140347830024848}%
If you have any information on how to fix this, I will update this section.\end{note}
%
\end{inlineexercise}
\begin{inlineexercise}{Images.}{x:exercise:check-images}%
Read more about images \href{https://pretextbook.org/doc/guide/html/overview-images.html}{here}.%
\par
In order to include an image in a document, make a \textless{}figure\textgreater{} tag and place a \textless{}caption\textgreater{} tag inside of it. If you want to have a description following the figure number, place that desciption inside of the \textless{}caption\textgreater{} tag. Next, make a tag \textless{}image source="images\slash{}name\textunderscore{}of\textunderscore{}image.png"\slash{}\textgreater{} (make sure the image is placed inside the "output\slash{}html\slash{}images\slash{}" directory of the document). If you want to have a description the line above the figure number, place the description in a \textless{}description\textgreater{} tag after the \textless{}image\slash{}\textgreater{} tag. \begin{example}{}{g:example:idp140347830438720}%
\textless{}figure\textgreater{}%
 \par
\textless{}caption\textgreater{}abc\textless{}\slash{}caption\textgreater{}%
 \par
\textless{}image source="images\slash{}name\textunderscore{}of\textunderscore{}image.png" width="60\%"\slash{}\textgreater{}%
 \par
\textless{}description\textgreater{}abc\textless{}\slash{}description\textgreater{}%
 \par
\textless{}\slash{}figure\textgreater{}%
%
\end{example}
 You can also specify the width in percent by placing width="NUMBER\%" after the location of the source inside of the \textless{}image\slash{}\textgreater{} tag as shown in the above example.%
\end{inlineexercise}
\begin{inlineexercise}{Imbedded Urls.}{x:exercise:check-imbedded-url}%
Read more about imbedded urls \href{https://pretextbook.org/doc/guide/html/topic-url.html}{here}.%
\par
Imbedding a url into a piece of text is similar in structure to the \textless{}image\slash{}\textgreater{} tag. You will use the tag \textless{}url href="https:\slash{}\slash{}link.com"\textgreater{}name of link inline\textless{}\slash{}url\textgreater{}.%
\par
The "name of the link inline" between the url tags will show up as blue text and will be underlined when you hover over it with your mouse.%
\end{inlineexercise}
\begin{inlineexercise}{Tables.}{x:exercise:check-tables}%
Read more about tables \href{https://pretextbook.org/doc/guide/html/topic-tabular.html}{here}.%
\par
Tables can be made by going to \href{}{LaTeX Table Editor} and plugging in your information. You can import your tables from Excel or LaTeX. Change the generate field from "LaTeX" to "PreTeXt" and copy the code that appears. \begin{note}{}{g:note:idp140347831594400}%
The website is a little glitchy and gets stuck performing certain funtions.\end{note}
%
\par
In order to make them "by-hand", you use the \textless{}table\textgreater{} tag (make sure to list an "xml:id" inside of the opening tag) and then create the number of rows inside here with the \textless{}row\textgreater{} tag. Then place the number of cells you want in each row with the \textless{}cell\textgreater{} tag. \mono{not complete, will finish in the future}%
\end{inlineexercise}
\end{subsectionptx}
%
%
\typeout{************************************************}
\typeout{Subsection 1.4.3 Quirks}
\typeout{************************************************}
%
\begin{subsectionptx}{Quirks}{}{Quirks}{}{}{x:subsection:sub-quirks-eli}
\begin{inlineexercise}{Subsection Issues.}{x:exercise:subsection-issues}%
One but\slash{}feature when using VSCode is that when you are adding a subsection, it seems to be required to add a introductiona and conclusion to that section. I don't know if it is a problem with other IDE's\slash{}editors but it will not compile with out it. Also, the introduction or conclusion cannot both be empty, you need to have something within at least one of them.%
\end{inlineexercise}
\begin{inlineexercise}{Problems with "\textbackslash{}sech" and "\textbackslash{}csch".}{x:exercise:sech-csch}%
When in math mode, for some reason \textbackslash{}sech and \textbackslash{}csch do not work and you have to enter them a different way. What worked for me so far is using \textbackslash{}mathrm\textbraceleft{}sech\textbraceright{} and \textbackslash{}mathrm\textbraceleft{}csch\textbraceright{} to make the text to display correctly.%
\end{inlineexercise}
\end{subsectionptx}
%
%
\typeout{************************************************}
\typeout{Subsection 1.4.4 GitHub.io}
\typeout{************************************************}
%
\begin{subsectionptx}{GitHub.io}{}{GitHub.io}{}{}{x:subsection:sub-githubio}
\begin{inlineexercise}{Setup.}{x:exercise:check-github_io-setup}%
Unfortunately I cannot (at this time) find the website that I used as a guide to build my GitHub.io site. Here is a somewhat similar \href{https://medium.com/@svinkle/publish-and-share-your-own-website-for-free-with-github-2eff049a1cb5}{\emph{guide}} on how to set up the inital site. After you after you do that, create a file named 'index.html'. You can use this as a general template for your index.html: \begin{example}{Index Template.}{x:example:template-index}%
\textless{}!DOCTYPE html\textgreater{} \textless{}html\textgreater{} \textless{}head\textgreater{} \textless{}title\textgreater{}Your Name\textless{}\slash{}title\textgreater{}  \textless{}link rel="stylesheet" type="text\slash{}css" href="\slash{}css\slash{}main.css"\textgreater{} \textless{}\slash{}head\textgreater{} \textless{}body\textgreater{} \textless{}nav\textgreater{} \textless{}ul\textgreater{} \textless{}li\textgreater{}\textless{}a href="\slash{}"\textgreater{}Home\textless{}\slash{}a\textgreater{}\textless{}\slash{}li\textgreater{} \textless{}li\textgreater{}\textless{}a href="https:\slash{}\slash{}links.links"\textgreater{}links\textless{}\slash{}a\textgreater{}\textless{}\slash{}li\textgreater{} \textless{}\slash{}ul\textgreater{} \textless{}\slash{}nav\textgreater{} \textless{}div class="container"\textgreater{} \textless{}div class="blurb"\textgreater{} \textless{}h1\textgreater{}Greating\textless{}\slash{}h1\textgreater{} \textless{}p\textgreater{}Information about you.\textless{}\slash{}p\textgreater{} \textless{}\slash{}div\textgreater{} \textless{}\slash{}div\textgreater{} \textless{}footer\textgreater{} \textless{}ul\textgreater{} \textless{}li\textgreater{}\textless{}a href="mailto:email@email.com"\textgreater{}email\textless{}\slash{}a\textgreater{}\textless{}\slash{}li\textgreater{} \textless{}li\textgreater{}\textless{}a href="https:\slash{}your-website.link"\textgreater{}github\textless{}\slash{}a\textgreater{}\textless{}\slash{}li\textgreater{} \textless{}\slash{}ul\textgreater{} \textless{}\slash{}footer\textgreater{} \textless{}\slash{}body\textgreater{} \textless{}\slash{}html\textgreater{}%
\end{example}
%
%
\end{inlineexercise}
\begin{inlineexercise}{Styling.}{x:exercise:check-styling}%
You can use this as a general template for your main.css: \begin{example}{CSS Template.}{x:example:template-css}%
body \textbraceleft{} margin: 60px auto; width: 70\%; \textbraceright{} nav ul, footer ul \textbraceleft{} font-family:'Helvetica', 'Arial', 'Sans-Serif'; padding: 0px; list-style: none; font-weight: bold; \textbraceright{} nav ul li, footer ul li \textbraceleft{} display: inline; margin-right: 20px; \textbraceright{} a \textbraceleft{} text-decoration: none; color: \#999; \textbraceright{} a:hover \textbraceleft{} text-decoration: underline; \textbraceright{} h1 \textbraceleft{} font-size: 3em; font-family:'Helvetica', 'Arial', 'Sans-Serif'; \textbraceright{} p \textbraceleft{} font-size: 1.5em; line-height: 1.4em; color: \#333; \textbraceright{} footer \textbraceleft{} border-top: 1px solid \#d5d5d5; font-size: .8em; \textbraceright{} ul.posts \textbraceleft{} margin: 20px auto 40px; font-size: 1.5em; \textbraceright{} ul.posts li \textbraceleft{} list-style: none; \textbraceright{}%
\end{example}
%
\end{inlineexercise}
\begin{inlineexercise}{Adding Pretext Files.}{x:exercise:check-adding-pretext}%
If you want to be able to host a PreTeXt project on your GitHub.io, make a folder in the main folder with the same name as your pretext book. Then place the contents of your 'ouput\slash{}html' folder of the project inside of the folder you just created. If you want to create a link to this inside of your website. In the index.html template above, substitue \mono{https://links.links} with the name of your pretext book folder \mono{/my-book/}%
\end{inlineexercise}
\end{subsectionptx}
\begin{conclusion}{}%
\end{conclusion}%
\end{sectionptx}
\end{chapterptx}
\end{document}